\documentclass{article}
\usepackage[backend=bibtex,sorting=none,autocite=superscript,style=numeric]{biblatex}
\renewcommand{\cite}[1]{\autocite{#1}}
\bibliography{../Thesis.bib}

\usepackage{titling}
\setlength{\droptitle}{-.5in}

%\usepackage{draftwatermark}
%\SetWatermarkLightness{.9}
%\SetWatermarkScale{4}

\title{Regional variation in onset and offset of respiratory syncytial virus season in Connecticut}
\author{Author & Ben Artin, YSPH/YSM 2018}
\preauthor{\begin{center}\begin{tabular}{r@{: }l}}
\postauthor{\\ Thesis advisor & Daniel Weinberger, PhD \\ Thesis reader & Virginia Pitzer, ScD \end{tabular}\end{center}}

\begin{document}

\maketitle

\section{Study Aims}

I propose to analyze existing data about respiratory syncytial virus (RSV) infections among infants hospitalized in Connecticut between 1997 and 2013, with the goal of establishing an association between geographic subdivisions (by ZIP code) and RSV season onset and duration. Based on the results of that analysis, I intend to propose a refinement to the guidelines for RSV immunoprophylaxis within Connecticut, aimed at optimizing the use of RSV immunoprophylaxis without adversely affecting RSV infection outcomes.

\section{Hypothesis}

RSV season onset changepoint in ZIP code A $\neq$ RSV season onset changepoint in ZIP code B, for any ZIP codes A and B (A $\neq$ B) in Connecticut; similarly for season offset changepoint.

\section{Background \& Rationale}

RSV causes seasonal respiratory illness of varying severity. RSV infection in infants is often associated with more severe illness (including bronchiolitis and pneumonia), sometimes requiring hospitalization. Several groups of infants are at higher risk of severe illness, hospitalization, and complications; these groups include premature and young infants, infants with lung and heart disease, and immunosuppressed children. \cite{CentersforDiseaseControlandPrevention:368RENtm}

While no targeted treatment exists for RSV infection, preventative monoclonal antibody medication (palivizumab) is available. Due to its high cost (in thousands of USD per patient per year) and its inconvenience (requiring monthly healthcare visits during RSV season), palivizumab immunoprophylaxis is recommended only for high-risk infants, and only during RSV season. \cite{AmericanAcademyofPediatricsCommitteeonInfectiousDiseases:2014bj,AmericanAcademyofPediatricsCommitteeonInfectiousDiseasesandCommitteeonFetusandNewborn:2003ug}

With exception of Florida, these guidelines are based on analysis of RSV seasonality on state and national level. \cite{AmericanAcademyofPediatricsCommitteeonInfectiousDiseasesandCommitteeonFetusandNewborn:2003ug} However, recent research has revealed regional variation in RSV season patterns within Connecticut. \cite{Noveroske:y4fi3188} This variation is potentially large enough to guide recommendations regarding administration of palivizumab, and therefore lower healthcare burden of RSV immunoprophylaxis.

Since healthcare burden of RSV immunoprophylaxis is primarily tied to the number of doses administered, which is mainly determined by the duration and onset of RSV season, analysis of RSV season duration and onset is needed to optimize use of RSV immunopropylaxis in Connecticut.

Prior analysis of regional variation in RSV seasonality within Connecticut established geographic variability in timing of peak RSV incidence, but did not consider duration or onset of RSV season. \cite{Noveroske:y4fi3188} The analysis I propose is a novel analysis of the existing data.

\section{Methods}

This will be a cross-sectional study of children less than 2 years old infected RSV infection between 1997 and 2013 in Connecticut.

The study population will be sampled using ICD-10 data from hospital admission records of all hospitalizations in Connecticut between 1997 and 2013, as reported to State of Connecticut Office of Health Care Access. The size of this sample is approximately 10000.

Sampled subjects will be aggregated by ZIP code of residence, and for each ZIP code RSV season onset and offset will be estimated using changepoint analysis. Changepoint analysis will be performed in R. 

\section{MPH Competencies}

\begin{itemize}
	\item Demonstrate a knowledge base in the disciplines of biostatistics, chronic and infectious disease epidemiology, health systems, public policy, social and behavioral sciences, and environmental health.
	\item Apply basic research skills to specific public health problems in both group and individual settings, including the ability to define problems; construct, articulate and test hypotheses; draw conclusions; and communicate findings to a variety of audiences.
	\item Apply public health concepts, principles, and methodologies obtained through formal course work to actual problems experienced in the community or work environment.
	\item Describe and critically evaluate approaches for the prevention and control of infectious diseases and define the key issues to their effective use.
	\item Apply principles and concepts obtained through coursework to design and implement studies on the etiology, detection, prevention or control of infectious diseases in the laboratory and field.
\end{itemize}

\printbibliography
	
\end{document}

